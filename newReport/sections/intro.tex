\section{Introduction}
\label{sec:intro}
Natural language generation (NLG) is an important subtask of many applications that rely on human computer interaction. While, as the Chinese proverb says, a picture is worth a thousand words, there are a lot of situations where pictures, schematic diagrams, maps, charts and plots cannot communicate information as effectively and efficiently as text would. There are no rules for when graphics is better than text. However, by means of observation, an annotated picture would better describe a physical location, charts can show progress, schematic diagrams can describe flow and hierarchy. All of them would fail, though, at describing abstract concepts such as causality or advice, which are quite straightforward to describe in text. Another issue on which the type of representation might depend is the expertise of the user, although graphical representation is considered an easy-to-understand way of representing information, research in psychology \cite{Reitera} has shown that in many cases, a considerable amount of information might be required to understand graphical representation and novice users might be better off with text.

In the current paper, an application that advices the user on what to wear with regard to the weather, is proposed. As noted earlier, text is probably the best way to communicate advices. A possible approach to this would be to simply provide a structured text, such as ``Weather: Rainy, What to wear: Raincoat''. Even though this might be informative, it is not user friendly. A better way to convey the message, could be naturally generated text with a friendlier tone, or a combination of text and graphics. However, due to limited time, resources and scope of the work, we attempt to generate plain natural language tips.

The system is built based on the structure of the Treebank-Trained Statistical Generator (TTSG) \cite{Belz}, which is explained and described in Section \ref{sec:relwork}. By alternating between different parameters of the system, as well as by proposing a new approach for the final step of the method, we attempt to take it apart, and examine the influence of every step of the procedure in the final result.

\subsection{Contribution}
As mentioned above, the system is built based on the structure of the TTSG method. This particular method was created and tested on a different kind of application. Therefore in the scope of this paper, the method, along with its variations, is going to be evaluated on its performance in a very specific type of text generation, i.e. short and concise tips. Additionally, an entirely new generating method is proposed, used and evaluated for the results it produces for this application.

\subsection{Outline}
In Section~\ref{sec:relwork} we overview different attempts in the field of natural language generation concerning the actual procedure of generating text, as well as the evaluation of the text itself. In Section~\ref{sec:method} we describe our implementation details, discussing different practical aspects. In Section~\ref{sec:exps} we describe in detail our evaluation experiment for the different parameters that have been used and discuss the effectiveness of the methodology of evaluation. Finally, in Section~\ref{sec:summary} we discuss the strengths and weaknesses of the variations of the method in light of our application.